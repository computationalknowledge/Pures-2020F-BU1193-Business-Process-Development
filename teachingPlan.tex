\section * {BU1193 Teaching Plan (3 hours per week)}
\section * {Students examine the core business processes that are integral to the activities of a business. }
\subsection * {LEARNING OUTCOMES:}
\begin{enumerate}
    \item Enterprise Resource Planning systems
    \item The integration of transaction level processes
    \item the "Order-to-Cash", "Procure-to-Pay"
    \item the "Production" processes
    \item Financial and Controlling processes
    \item Human Resource processes are also examined. 
\end{enumerate}

\section{Course Topics}
\begin{itemize}
    \item Resource Planning Systems
    \item   Transaction Level Processes
    \item   Order to Cash
    \item   Procure to Pay
    \item   Production Processes
    \item   HR Processes
    \item  Business processes define the steps involved in completing various business activities, such as order taking, purchasing, materials management, financial accounting and planning. Understanding these processes and how they cross department and organizational boundaries is critical to assessing real world business operations. 
    \item   Order Taking
    \item   Purchasing
    \item   Materials Management
    \item   Financial Accounting
    \item   Planning
    \item Students enhance their knowledge of how computer systems support business processes through extensive hands-on experience using commercially available Enterprise Resource Planning (ERP) software.
\end{itemize}    
    
\section {  Course Objectives }
    1.    Assess core business processes including order-to-cash, procure-to-pay, and integrated production process.
    2.    Describe supporting business processes including Finance and Control and Human Resources.
    3.    Differentiate between a variety of business operations models including maketo-stock, make-to-order, engineer-to-order, and outsourcing.
    4.    Assess the impact of ‘master data’ and ‘transactional data’ requirements on how Enterprise Resource Planning (ERP) software supports an enterprise.
    5.    Assign typical activities in a business process to functional areas within an enterprise.
    5.    Assign typical activities in a business process to functional areas within an enterprise.
    6.    Simulate business transactions through entire business processes using an ERP system.
    7.    Predict relevant impacts on business operations resulting from a change in a process.
    8.    Apply basic knowledge of project management concepts and systems life cycle methodologies to information technology in business environments.
    9.    Communicate information technology approaches accurately, persuasively, and credibly to internal and external clients.
  
\section {Day Teaching Plans} 
\begin{itemize} 
\item Class 01 Sept14
      Course Introduction, Grading Scheme, Class Rules and Operations Details
      On Week 01, we introduce the study of Enterprise Resource Planning systems, the integration of transaction level processes and the "Order-to-Cash", "Procure-to-Pay", and the "Production" processes. Financial and Controlling processes as well as Human Resource processes are also examined. 
        Resource Planning Systems
        Transation Level Processes
        Order to Cash
        Procure to Pay
        Production Processes
        HR Processes
         Business processes define the steps involved in completing various business activities, such as order taking, purchasing, materials management, financial accounting and planning. Understanding these processes and how they cross department and organizational boundaries is critical to assessing real world business operations. 
          Order Taking
          Purchasing
          Materials Management
          Financial Accounting
          Planning
\item  Class 02 Sept 21 

 
      \item Assess core business processes including order-to-cash, procure-to-pay, and integrated production process.
    Class 03 Sept 28
      2.    Describe supporting business processes including Finance and Control and Human Resources.
    Class 04 Oct 05
      3.    Differentiate between a variety of business operations models including maketo-stock, make-to-order, engineer-to-order, and outsourcing.
    Class 05 Oct 12
      Quiz 1  10%
      Assignment 1 DUE  10%
        Based on Learning Outcomes 1, 2, 3
      4.    Assess the impact of ‘master data’ and ‘transactional data’ requirements on how Enterprise Resource Planning (ERP) software supports an enterprise.
    Class 06 Oct 19
      5.    Assign typical activities in a business process to functional areas within an enterprise.
    Class 07 Oct 26
      6.    Simulate business transactions through entire business processes using an ERP system.
    Class 08 Nov 02
      7.    Predict relevant impacts on business operations resulting from a change in a process.
    Class 09 Nov 09
      Quiz 2  10%
      Assignment 2 DUE 10%
      Review and Class Exercises: Prep for Final Exam
    Class 10 Nov 16
      8.    Apply basic knowledge of project management concepts and systems life cycle methodologies to information technology in business environments.
    Class 11 Nov 23
      9.    Communicate information technology approaches accurately, persuasively, and credibly to internal and external clients.
    Class 12 Nov 30
      Project  DUE     20%
        Students use Excel to build an ERP System that illustrates all the Topics we have studied
      Review and Class Exercises: Prep for Final Exam
    Class 13 Dec 07 Final Exam Week
    \end{itemize}
\section * {   Grading Scheme }
    Participation       10%
      Will be taken in the Format of quizzes held at random times during the classes
    Project      20%
      Students will use Excel to build an ERP System that illustrates all the Topics we have studied
    Assignment 1  10%
    Assignment 2 10%
    Quiz 1  10%
    Quiz 2  10%
    Final Exam       30%
\section * {  Some of the Technologies we will be using:}
    Firebase
    Excel
    Power BI
    Google Sites, Google Docs, Google Apps