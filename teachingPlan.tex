\section * {BU1193 Teaching Plan (3 hours per week)}
\section * {Course Visual Learning Roadmap}
\url{https://t2m.io/ERPRoadMap}
\section * {Course purpose and objectives: }

Students examine the core business processes that are integral to the activities and operation of a the modern business operation.

\subsection * {Because many students in this class will have not had deep prior experience with managing a business, we will make this more fun and engaging by constructing an enterprise resource planning database in this course.}  
 
By the end of this course each one of you will have built a working ERP database providing on a small scale the same core functionalities that SAP delivers.
\newline
In this way you will have actually been responsible for providing a data model and a set of processes that could be used to run an actual real company.
\newline
This is something to get very excited about!

\subsection * {LEARNING OUTCOMES:}
\begin{enumerate}
    \item Enterprise Resource Planning systems
    \item The integration of transaction level processes
    \item the "Order-to-Cash", "Procure-to-Pay"
    \item the "Production" processes
    \item Financial and Controlling processes
    \item Human Resource processes are also examined. 
\end{enumerate}

\section{Course Topics}
\begin{itemize}
    \item Resource Planning Systems
    \item   Transaction Level Processes
    \item   Order to Cash
    \item   Procure to Pay
    \item   Production Processes
    \item   HR Processes
    \item  Business processes define the steps involved in completing various business activities, such as order taking, purchasing, materials management, financial accounting and planning. Understanding these processes and how they cross department and organizational boundaries is critical to assessing real world business operations. 
    \item   Order Taking
    \item   Purchasing
    \item   Materials Management
    \item   Financial Accounting
    \item   Planning
    \item Students enhance their knowledge of how computer systems support business processes through extensive hands-on experience using commercially available Enterprise Resource Planning (ERP) software.
\end{itemize}    
    \newpage
\section {Course Objectives }
\begin{enumerate}
    \item   Assess core business processes including order-to-cash, procure-to-pay, and integrated production process.
    \item   Describe supporting business processes including Finance and Control and Human Resources.
    \item   Differentiate between a variety of business operations models including make-to-stock, make-to-order, engineer-to-order, and outsourcing.
    \item   Assess the impact of ‘master data’ and ‘transactional data’ requirements on how Enterprise Resource Planning (ERP) software supports an enterprise.
    \item   Assign typical activities in a business process to functional areas within an enterprise.
    \item   Assign typical activities in a business process to functional areas within an enterprise.
    \item   Simulate business transactions through entire business processes using an ERP system.
    \item   Predict relevant impacts on business operations resulting from a change in a process.
    \item   Apply basic knowledge of project management concepts and systems life cycle methodologies to information technology in business environments.
    \item   Communicate information technology approaches accurately, persuasively, and credibly to internal and external clients.
\end{enumerate}

\section {Day Teaching Plans} 

\subsection {Class 01 September 14 }  
 
      Course Introduction, Grading Scheme, Class Rules and Operations Details
      \\
      \subsection{On Week 01, we introduce  }
      \begin{itemize}
          \item the study of Enterprise Resource Planning systems
          \item the integration of transaction level processes          
          \item "Order-to-Cash" process
          \item "Procure-to-Pay" process
          \item "Production"  process        
          \item Financial processes    
          \item Controlling processes    
          \item Human Resource processes          
    
      \end{itemize}

      \begin{enumerate}
          \item Resource Planning Systems
          \item Transaction Level Processes          
          \item Order to Cash          
          \item Procure to Pay    
          \item Production Processes              
          \item HR Processes              
      \end{enumerate}   
      
      \subsection * {Business processes }
         Business processes define the steps involved in completing various business activities, such as order taking, purchasing, materials management, financial accounting and planning. Understanding these processes and how they cross department and organizational boundaries is critical to assessing real world business operations.       
      \begin{itemize}
          \item Order Taking
          \item Purchasing
          \item Materials Management          
          \item Financial Accounting          
          \item Planning          
      \end{itemize}        

\subsection{Class 02 September 21 }
      Assess core business processes including order-to-cash, procure-to-pay, and integrated production process.

\subsection{ Class 03 September 28 }
    Describe supporting business processes including Finance and Control and Human Resources.
\subsection{Class 04 October 05}
      Differentiate between a variety of business operations models including maketo-stock, make-to-order, engineer-to-order, and outsourcing.
\subsection{    Class 05 Oct 12}
      Quiz 1  10\%   \\
      Assignment 1 DUE  10\%    
        Based on Learning Outcomes 1, 2, 3   \\ 
        \newline
       Assess the impact of ‘master data’ and ‘transactional data’ requirements on how Enterprise Resource Planning (ERP) software supports an enterprise.
\subsection{    Class 06 Oct 19}
      Assign typical activities in a business process to functional areas within an enterprise.
\subsection{    Class 07 Oct 26}
      Simulate business transactions through entire business processes using an ERP system.
\subsection{    Class 08 Nov 02}
      Predict relevant impacts on business operations resulting from a change in a process.
\subsection{    Class 09 Nov 09}
     Quiz 2  10%
    \\  Assignment 2 DUE 10%
    \\  Review and Class Exercises: Prep for Final Exam
\subsection{    Class 10 Nov 16}
      Apply basic knowledge of project management concepts and systems life cycle methodologies to information technology in business environments.
\subsection{    Class 11 Nov 23}
      Communicate information technology approaches accurately, persuasively, and credibly to internal and external clients.
\subsection{    Class 12 Nov 30}
\begin{itemize}
    \item  Project  DUE     20\%   \\ Students use Excel to build an ERP System that illustrates all the Topics we have studied\newline
    \item    Review and Class Exercises: Prep for Final Exam  \\
\end{itemize}      
      
\subsection{Class 13 Dec 07 Final Exam Week}
Review of all course Topics \\
Final Exam



\section * {  Some of the Technologies we will be using:}
\begin{itemize}
    \item Firebase
    \item Excel    
    \item Power BI  
    \item Google Sites, Google Docs, Google Apps    
\end{itemize}
    
    
    
    